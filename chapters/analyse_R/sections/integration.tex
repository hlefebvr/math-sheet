\subsection{Définitions}
\begin{description}
\item[Fonction en escalier] : Fonctions constantes sur des intervalles
\item[Intégrale de Riemann] : Soit 
    \[ f=\sum_{i=1}^n \alpha_i1_{I_i} \]
    une fonction en escalier,
    on définit l'intégrale de $f$ par
    \[
        I(f)=\int_a^bf(t)dt=\sum\alpha_i(x_{i+1}-x_i)
    \]
    Pour une fonction quelconque, s’il existe, pour tout $\varepsilon > 0$, deux fonctions en escalier $f_\varepsilon$ et $F_\varepsilon$ telle que
    $f_\varepsilon \le f \le F_\varepsilon$
    et
    $I(f_\varepsilon ) - I(f_\varepsilon ) < \varepsilon)$, alors $f$ est dite Riemann-intégrable et on a :
    \[
        \int_a^bf(t)dt=
        \sup\left\{
        I(g) | g\textrm{ fonction en escalier et }
        g \le f    
        \right\}
    \]
\item[Fonction étagée] : Fonction dont l’image est constituée d’un nombre fini de valeurs réelles
\item[Théorème] : Toute fonction à valeur dans $\R^n$ est limite de fonctions étagées
\item[Intégrale de Lebesgue] : Soit
    \[ f=\sum_{i=1}^n \alpha_i1_{A_i} \]
    une fonction étagée, on définit l’intégrale de $f$ par rapport à la mesure $\mu$ par
    \[
        \int_Xfd\mu=\sum_{i=1}^n\alpha_i\mu(A_i)
    \]
    et pour $E\subset X$
    \[ \int_Efd\mu=\int_Xf1_Ed\mu \]
    Pour $f$ une fonction positive,
    \[
        \int_Xfd\mu=\sup\left\{
            \int sd\mu|s\textrm{ étagée et } s\le f
        \right\}
    \]
    Enfin pour une fonction quelconque, on définit :
    $f^+ = \max(0, f)$ et $f^- = \max(0,-f)$ de sorte que :
    \[ \int fd\mu=\int f^+d\mu+\int f^-d\mu \]

\end{description}
\subsection{Propriétés}
\begin{description}
\item[Lien Riemann-Lebesgue] : Si $f$ est Riemann-Intégrable, alors $f$ est Lebesgue-intégrable
\item[Ensemble de fonctions intégrables] (au sens de Lebesgue) :
    \[
        L^p(A) = \left\{
            f:\R\rightarrow\R |
            \int_A |f|^p<\infty    
        \right\}
    \]
\item[Fonctions localement intégrables] : $f:\R\rightarrow\R$ Lebesgue-intégrable sur tout intervalle borné ($L^1\subset L^1_{loc}$)
\item[Intégration et dérivation] :
    \[
        f(x)=\limite{h}{0}
        \frac{1}{h}
        \int_x^{x+h}f(t)dt
    \]
\item[Egalité d’intégrales] : 
    \[
        f\overset{pp}{=}g\Leftrightarrow
        \int f(t)dt=\int g(t)dt
    \]
\item[Linéarité] : 
    \[ \int (f(t)+\lambda g(t))dt=\int f(t)dt + \lambda\int g(t)dt \]
\item[Relation de Chasles] : Qui implique aussi $\int_a^bf(t)dt=-\int_b^af(t)dt$
    \[ \int_a^bf(t)dt=\int_a^cf(t)dt+\int_c^bf(t)dt \]
\item[Relation d'ordre] : 
    \[ f\le g\Leftrightarrow\int f(t)dt\le\int g(t)dt \]
\item[Fonction périodique] : Soit $f$ une fonction $T$-périodique,
    \[ \int_0^Tf(t)dt=\int_c^{c+T}f(t)dt \]
\item[Inégalité triangulaire] :
    \[ \left|\int f(t)dt\right|\le\int|f(t)|dt \]
\item[Cauchy-Schwartz] : 
    \[
        \left|
        \int f(t)g(t)dt
        \right|\le
        \sqrt{
            \int f^2(t)dt
            \times
            \int g^2(t)dt
        }
    \]
\item[Inégalité de Holder] :
    \[
        \frac{1}{p}+\frac{1}{q}=1
        \Rightarrow
        \int f(t)g(t)dt \le
        \left( \int |f(t)|^pdt \right)^{\frac{1}{p}}
        \left( \int |g(t)|^qdt \right)^{\frac{1}{q}}
    \]
\item[Théorème de la moyenne] :
    \[
        \forall x\in[a,b], m\le f\le M,\Rightarrow
        m\le\frac{1}{b-a}\int_a^bf(t)dt\le M
    \]
\item[Inégalité de la moyenne] :
    \[
        \left|
            \int_a^bf(x)g(x)dx
        \right|
        \le
        \sup_{x\in[a,b]} |f(x)|\times
        \int_a^b|g(x)|dx
    \]
\item[Intégrale sur un ensemble négligable] : Soit $\mu$ une mesure alors
    \[ \mu(E)=0\Rightarrow\int_Efd\mu=0 \]
\item[Théorème fondamental] :
    \[
        f(x)=f(a)+\int_a^xf'(t)dt
    \]
\item[Intégration par partie] (IPP) : 
    \[
        \int_a^bu'(t)v(t)dt=
        [u(t)v(t)]_a^b-
        \int_a^bu(t)v'(t)dt
    \]
\item[Changement de variable] : 
    \[
        \int_a^bf(x)dx
        \overset{x=u(t)}{=}
        \int_{u^{-1}(a)}^{u^{-1}(b)}
        f(u(t))u'(t)dt
    \]
\item[Propositions sur l’intégrabilité] :
    \begin{itemize}
    \item $f$ monotone $\Rightarrow f$ Riemann-intégrable
    \item $f$ continue $\Rightarrow f$ Riemann-intégrable
    \item $f$ pp-continue et bornée $\Rightarrow f$ Riemann-intégrable
    \item $f$ pp-continue $\Rightarrow f$ Lebesgue-intégrable
    \item $|f | < g$, $g$ Lebesgue-intégrable $\Rightarrow f$ Lebesgue-intégrable
    \item $f$ Lebesgue-intégrable $\Leftrightarrow |f|$ Lebesgue-intégrable
    \end{itemize}
\end{description}
\subsection{Convergence}
\begin{description}
\item[Convergence] (Riemann) : 
    \[ f_n\overset{unif}\longrightarrow f
    \Rightarrow
    \int f_n(t)dt \overset{unif}\longrightarrow \int f(t)dt \]
\item[Théorème de convergence monotone] (Beppo-Levi) : 
    \[
        \begin{cases}
            (f_n)\textrm{ suite croissante de fonction}\\
            f_n\longrightarrow f, n\longrightarrow\infty
        \end{cases}
        \Leftrightarrow
        \int f_n\longrightarrow\int f, n\longrightarrow\infty
    \]
\item[Théorème de convergence dominée] : 
    \[
        \begin{cases}
            f_n\overset{pp}{\longrightarrow}f\\
            |f_n|< g, g\in L^1
        \end{cases}
        \Rightarrow
        \int f_n\longrightarrow\int f 
        \left(
        \textrm{et même : }
        \int|f_n-f|\longrightarrow 0    
        \right)
    \]
\item[Inversion somme-integrale] : 
    \[
        (f_n) \textrm{ suite de fonction positive}
        \Rightarrow
        \int\sum^\infty f_n(x)dx=\sum^\infty\int f_n(x)dx
    \]
\item[Théorème de Fubini] : 
    \[
        f\in L^1\Rightarrow
        \iint f(x,y)dxdy=\int\left(\int f(x,y)dx\right)dy
    \]
\item[Théorème de Fubini-Tonnelle] : 
    \[
        f\ge 0\Rightarrow
        \iint f(x,y)dxdy=\int\left(\int f(x,y)dx\right)dy
    \]
\item[Définition] : intégrale de fonction discontinue, intégrale sur un intervalle non bornée, etc.
\item[Intégrales Riemann-impropre de références] :
    \[
        \int_0^1\frac{dt}{t^\alpha}\textrm{ converge si }\alpha <1
        \textrm{ ; }
        \int_1^\infty\frac{dt}{t^\alpha}\textrm{ converge si }\alpha >1
        \textrm{ ; }
        \int_0^1\ln tdt=-1
    \]
\item[Riemann-impropre et Lebesgue] : Si $f$ est Riemann-intégrable au sens impropre et de signe constant alors $f$ est
Lebesgue-intégrable
\end{description}
\subsection{Intégrale de Riemann-Stieltjes}
\begin{description}
\item[Définition] : Si $\alpha$ est une fonction croissante, alors elle définit une mesure.
    On appelle intégrale de Riemann-Stieltjes l’intégrale par rapport à cette mesure :
    $\int f(x)d\alpha(x)$
    et on a :
    \[
    \begin{split}
        \alpha([a, b]) = \alpha(b^+)-\alpha(a^-)\\
        \alpha([a, b[) = \alpha(b^-)-\alpha(a^-)\\
        \alpha(]a, b[) = \alpha(b^-)-\alpha(a^+)\\
        \alpha(]a, b]) = \alpha(b^+)-\alpha(a^+)
    \end{split}
    \]
\item[Calcul] : 
    \[ \int f(x)d\alpha(x)=\int f(x)\alpha'(x)dx \]
\end{description}
\subsection{Fonctions définies par une intégrale}
\begin{description}
\item[Définition] : Soit $f:(x, t)\rightarrow f(x, t)$, si $f$ est continue en $t$ pour presque-tout $x$
    et $|f(t, x)| \le g(x), g \in L^1$
    alors la fonction suivante est défini et est continue
    \[ F(t)=\int f(t,x)dx \]
\item[Dérivabilité] : Si $\dpartial{f}{t}(x,t)$ existe et est continue et
    $\left|\dpartial{f}{x}(x,t)\right|<g(x),g\in L^1$ alors $F$ est dérivable et 
    \[ \frac{dF}{dt}(t)=\int \dpartial{f}{t}(t,x)dx \]
\item[Formule] : 
    \[ F(t)=\int_{[u(t),v(t)]}f(x,t)dx \]
    \[
        \frac{dF}{dt}(t)=
        f(t,v(t))\frac{dv(t)}{dt}+
        f(t,u(t))\frac{du(t)}{dt}+
        \int_{[u(t),v(t)]}\dpartial{f}{t}(x,t)dx
    \]
\end{description}
\subsection{Introduction au calcul des variations}
\begin{description}
\item[Problème de variation] : Trouver $u^*$ telle que
    \[
        u^*=\min_{u\in K}J(u)
        \textrm{ avec }
        J(u)=\int_\alpha^\beta\varphi(u,\dot u, t)dt
    \]
\item[Équation d'Euler-Lagrange] : $u$ solution du problème de variation, alors
    \[
        \dpartial{}{u}\varphi(u,\dot u,t)-
        \frac{d}{dt}\left[\dpartial{}{\dot u}\varphi(u,\dot u, t)\right]=0
    \]
\item[Intégrale première d’Euler-Lagrange] : $\varphi(u,\dot u,t)=\varphi(u, \dot u)$
    \[
        \varphi(u,\dot u)=\left[\dpartial{}{\dot u}\varphi(u,\dot u)\right]\dot u+k, k\in\R
    \]
\item[Condition aux limites] :
    \begin{itemize}
    \item Deux extrémités fixes : $u(\alpha)=a$ et $u(\beta)=b$
    \item Une extrémité libre : $u(\alpha)=a$ et $\dpartial{}{\dot u}\varphi(u(\beta),\dot u(\beta), \beta)=0$
    \item Deux extrémités libres : $\dpartial{}{\dot u}\varphi(u(\alpha),\dot u(\alpha), \alpha)=0$ et $\dpartial{}{\dot u}\varphi(u(\beta),\dot u(\beta), \beta)=0$
    \end{itemize}
\end{description}