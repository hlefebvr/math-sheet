\subsection{Définitions}
\begin{description}
\item[Fonction en escalier] : Fonctions constantes sur des intervalles
\item[Intégrale de Riemann] : Soit 
    \[ f=\sum_{i=1}^n \alpha_i1_{I_i} \]
    une fonction en escalier,
    on définit l'intégrale de $f$ par
    \[
        I(f)=\int_a^bf(t)dt=\sum\alpha_i(x_{i+1}-x_i)
    \]
    Pour une fonction quelconque, s’il existe, pour tout $\varepsilon > 0$, deux fonctions en escalier $f_\varepsilon$ et $F_\varepsilon$ telle que
    $f_\varepsilon \le f \le F_\varepsilon$
    et
    $I(f_\varepsilon ) - I(f_\varepsilon ) < \varepsilon)$, alors $f$ est dite Riemann-intégrable et on a :
    \[
        \int_a^bf(t)dt=
        \sup\left\{
        I(g) | g\textrm{ fonction en escalier et }
        g \le f    
        \right\}
    \]
\item[Fonction étagée] : Fonction dont l’image est constituée d’un nombre fini de valeurs réelles
\item[Théorème] : Toute fonction à valeur dans $\R^n$ est limite de fonctions étagées
\item[Intégrale de Lebesgue] : Soit
    \[ f=\sum_{i=1}^n \alpha_i1_{A_i} \]
    une fonction étagée, on définit l’intégrale de $f$ par rapport à la mesure $\mu$ par
    \[
        \int_Xfd\mu=\sum_{i=1}^n\alpha_i\mu(A_i)
    \]
    et pour $E\subset X$
    \[ \int_Efd\mu=\int_Xf1_Ed\mu \]
    Pour $f$ une fonction positive,
    \[
        \int_Xfd\mu=\sup\left\{
            \int sd\mu|s\textrm{ étagée et } s\le f
        \right\}
    \]
    Enfin pour une fonction quelconque, on définit :
    $f^+ = \max(0, f)$ et $f^- = \max(0,-f)$ de sorte que :
    \[ \int fd\mu=\int f^+d\mu+\int f^-d\mu \]

\end{description}
\subsection{Propriétés}
\begin{description}
\item[Lien Riemann-Lebesgue] : Si $f$ est Riemann-Intégrable, alors $f$ est Lebesgue-intégrable
\item[Ensemble de fonctions intégrables] (au sens de Lebesgue) :
    \[
        L^p(A) = \left\{
            f:\R\rightarrow\R |
            \int_A |f|^p<\infty    
        \right\}
    \]
\item[Fonctions localement intégrables] : $f:\R\rightarrow\R$ Lebesgue-intégrable sur tout intervalle borné ($L^1\subset L^1_{loc}$)
\item[Intégration et dérivation] :
    \[
        f(x)=\limite{h}{0}
        \frac{1}{h}
        \int_x^{x+h}f(t)dt
    \]
\item[Egalité d’intégrales] : 
    \[
        f\overset{pp}{=}g\Leftrightarrow
        \int f(t)dt=\int g(t)dt
    \]
\item[Linéarité] : 
    \[ \int (f(t)+\lambda g(t))dt=\int f(t)dt + \lambda\int g(t)dt \]
\item[Relation de Chasles] : Qui implique aussi $\int_a^bf(t)dt=-\int_b^af(t)dt$
    \[ \int_a^bf(t)dt=\int_a^cf(t)dt+\int_c^bf(t)dt \]
\item[Relation d'ordre] : 
    \[ f\le g\Leftrightarrow\int f(t)dt\le\int g(t)dt \]
\item[Fonction périodique] : Soit $f$ une fonction $T$-périodique,
    \[ \int_0^Tf(t)dt=\int_c^{c+T}f(t)dt \]
\item[Inégalité triangulaire] :
    \[ \left|\int f(t)dt\right|\le\int|f(t)|dt \]
\item[Cauchy-Schwartz] : 
    \[
        \left|
        \int f(t)g(t)dt
        \right|\le
        \sqrt{
            \int f^2(t)dt
            \times
            \int g^2(t)dt
        }
    \]
\item[Inégalité de Holder] :
    \[
        \frac{1}{p}+\frac{1}{q}=1
        \Rightarrow
        \int f(t)g(t)dt \le
        \left( \int |f(t)|^pdt \right)^{\frac{1}{p}}
        \left( \int |g(t)|^qdt \right)^{\frac{1}{q}}
    \]
\item[Théorème de la moyenne] :
    \[
        \forall x\in[a,b], m\le f\le M,\Rightarrow
        m\le\frac{1}{b-a}\int_a^bf(t)dt\le M
    \]
\item[Inégalité de la moyenne] :
    \[
        \left|
            \int_a^bf(x)g(x)dx
        \right|
        \le
        \sup_{x\in[a,b]} |f(x)|\times
        \int_a^b|g(x)|dx
    \]
\item[Intégrale sur un ensemble négligable] : Soit $\mu$ une mesure alors
    \[ \mu(E)=0\Rightarrow\int_Efd\mu=0 \]
\item[Théorème fondamental] :
    \[
        f(x)=f(a)+\int_a^xf'(t)dt
    \]
\item[Intégration par partie] (IPP) : 
    \[
        \int_a^bu'(t)v(t)dt=
        [u(t)v(t)]_a^b-
        \int_a^bu(t)v'(t)dt
    \]
\item[Changement de variable] : 
    \[
        \int_a^bf(x)dx
        \overset{x=u(t)}{=}
        \int_{u^{-1}(a)}^{u^{-1}(b)}
        f(u(t))u'(t)dt
    \]
\item[Propositions sur l’intégrabilité] :
    \begin{itemize}
    \item $f$ monotone $\Rightarrow f$ Riemann-intégrable
    \item $f$ continue $\Rightarrow f$ Riemann-intégrable
    \item $f$ pp-continue et bornée $\Rightarrow f$ Riemann-intégrable
    \item $f$ pp-continue $\Rightarrow f$ Lebesgue-intégrable
    \item $|f | < g$, $g$ Lebesgue-intégrable $\Rightarrow f$ Lebesgue-intégrable
    \item $f$ Lebesgue-intégrable $\Leftrightarrow |f|$ Lebesgue-intégrable
    \end{itemize}
\end{description}
\subsection{Convergence}
\begin{description}
\item[]
\end{description}
\subsection{Fonctions définies par une intégrale}
\begin{description}
\item[]
\end{description}
\subsection{Introduction au calcul des variations}
\begin{description}
\item[]
\end{description}