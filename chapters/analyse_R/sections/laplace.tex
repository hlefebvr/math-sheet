\subsection{Fonctions}
\begin{description}
\item[Définition] :
    \[
        \tilde f(s)=\mathcal Lf(s)=\int_0^\infty f(x)e^{-sx}dx
    \]
\item[Théorème] : $\tilde f$ est holomorphe et
    \[
        \frac{d^k}{ds^k}\tilde f(s)=
        \int_0^\infty f(x)(-x)^ke^{-sx}dx, \forall k\in\N
    \]
\item[Théorème] : Si $F$ est une fonction analytique dans le demi-plan complexe ${z\in\C | Re(z) > \eta 0 }$, et si, en tant
    que fonction de $\eta = Im(z)$, $F$ est intégrable, alors elle est la transformée de Laplace d’une fonction continue
    telle que
    \[
        f(x)=\frac{1}{2i\pi}\int_{\xi-i\infty}^{\xi+i\infty}f(x)e^{zx}dz
    \]
\item[Théorème] : Si les transformées de Laplace coïncides pour un $Re(s)$ assez grand alors $f = g$
\item[Exemples] :
\begin{multicols}{2}
    \begin{enumerate}
        \item\[
            \widetilde{Y(x)x^a}=\frac{\Gamma(a+1)}{s^{a+1}}
        \]
        \item\[
            \widetilde{Y(x)e^{ax}}=\frac{1}{s-a}
        \]
    \end{enumerate}
\end{multicols}
\item[Propriétés] :
\begin{multicols}{2}
    \begin{enumerate}
        \item\[
            \mathcal L(e^{-at}f(t))=\tilde(s+a)
        \]
        \item\[
            \mathcal L(f^{(n)}(t))=s^n\tilde f(s)-s^{n-1}f(0)-\dots-f^{(n-1)}(0)
        \]
        \item\[
            \mathcal L\left(\int_0^tf(u)du\right)=\frac{\tilde f(s)}{s}
        \]
        \item\[
            \mathcal L(tf(t))=-\tilde f'(s)
        \]
        \item\[
            \mathcal L\left(\frac{f(t)}{t}\right)=\int_0^s\tilde f(p)dp
        \]
        \item\[
            \mathcal L(f*g)=\tilde f.\tilde g
        \]
        \item Si $f$ est $T$-périodique, alors\[
            \mathcal L(f)(s)=\frac{\int_0^Tf(t)e^{-st}dt}{1-e^{-st}}
        \]
    \end{enumerate}
\end{multicols}
\item[Transformée inverse] :
\begin{enumerate}
    \item Linéarité : $\mathcal L^{-1}(a\tilde f+b\tilde g)=a\mathcal L^{-1}(\tilde f)+b\mathcal L^{-1}(\tilde g)=af+bg$
    \item Translation : $\mathcal L^{-1}(\tilde f(s-a))=e^{at}f(t)$
    \item Modulation : $\mathcal L^{-1}(e^{-as}\tilde f(s))=\begin{cases}f(t-a),\textrm{ si }t>a\\ 0,\textrm{ sinon}\end{cases}$
    \item Changement d'échelle : $\mathcal L^{-1}(\tilde f(ks))=\dfrac{1}{k}f\left(\dfrac{1}{k}\right)$
    \item Dérivée : $\mathcal L^{-1}(\tilde f^{(k)}(s))=(-1)^kt^kf(t)$
    \item Intégrale : $\mathcal L^{-1}\left(\int_0^\infty\tilde f(s)ds\right)=\dfrac{f(t)}{t}Y(t)$
    \item Multiplication par $s$ : $\mathcal L^{-1}(sf(s))=f'(t)+f(0)\delta$
\end{enumerate}
\item[Théorèmes taubériens] :
    \begin{align*}
        \limite{t}{0}f(t)&=\limite{s}{\infty}s\tilde f(s)\\
        \limite{t}{\infty}f(t)&=\limite{s}{0}s\tilde f(s)
    \end{align*}
\end{description}
\subsection{Distributions}
\begin{description}
\item[Définition] : $T\in\D'_+$
    \[
        \mathcal L(T)=\tilde T=<T,e^{-st}>
    \]
\begin{multicols}{2}
\item[Exemples] :
    \begin{enumerate}
        \item $\tilde\delta=1$
        \item $\tilde\delta_a=e^{-as}$
        \item $\widetilde{\delta'}=s$
        \item $\widetilde{\delta^{(n)}}=s^n$
    \end{enumerate}
\end{multicols}
\end{description}