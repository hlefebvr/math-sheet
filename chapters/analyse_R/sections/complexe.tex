\begin{description}
\item[Définition] : 
    \[ \C=\{a+ib | a,b\in\R\textrm{ et } i^2=-1 \} \]
\item[Partie réelle et imaginaire] :
    \[ Re(a+ib)=a\textrm{ et }Im(a+ib)=b \]
\item[Module et argument] : 
    \[
        |z|=\sqrt{a^2+b^2}
        \textrm{ et }
        \arg z = \tan\frac{b}{a}
    \]
\item[Écriture d'un nombre complexe] :
    $\forall z\in\C, \exists a,b,r,\theta\in\R\tq$
    \[
        z=a+ib
        =re^{i\theta}
        =r(\cos\theta+i\sin\theta)
        \textrm{ avec }
        r=|z|
        \textrm{ et }
        \theta = \arg z
    \]
\item[Conjugaison] : Soi $z=a+ib$ alors $\bar z=a-ib$ et 
    \[
        \overline{z_1+z_2}=\overline{z_1}+\overline{z_2}
        \textrm{ ; }
        \overline{z_1.z_2}=\overline{z_1}.\overline{z_2}
        \textrm{ ; }
        \overline{\left(\frac{1}{z}\right)}=\frac{1}{\overline z}
        \textrm{ ; }
        \overline{\overline{z}}=z
    \]
    \[
        z+\overline{z}=2\times Re(z)
        \textrm{ ; }
        z-\overline{z}=2i\times Im(z)
    \]
\item[Calcul avec les modules] :
    \[
        z\bar z=|z|^2
        \textrm{ ; }
        \left|\frac{z_1}{z_2}\right|=\frac{|z_1|}{|z_2|}
        \textrm{ ; }
        |z_1z_2|=|z_1||z_2|
        \textrm{ ; }
        |z|=|\bar z|
    \]
\item[Calcul avec les arguments] :
    \[
        \arg(z_1z_2)=\arg(z_1)+\arg(z_2)[2\pi]
        \textrm{ ; }
        \arg\left(\frac{1}{z}\right)=-\arg(z)
    \]
\item[Formule de Moivre] : 
    \[ (\cos\theta+i\sin\theta)^n=\cos n\theta+i\sin n\theta \]
\item[Formules d'Euler] : 
    \[
        \cos x=\frac{e^{ix}+e^{-ix}}{2}
        \textrm{ ; }
        \sin x=\frac{e^{ix}-e^{-ix}}{2i}
    \]
\item[Théorème de D’Alembert-Gauss] : Toute équation algébrique de $\C$ admet au moins une solution dans $\C$
\item[Racine $n$-ième] :
    \[
        z^n=\alpha\Leftrightarrow
        \begin{cases}
            |z|=|\alpha|^{\frac{1}{n}}\\
            \arg z=\frac{\arg\alpha}{n}+\frac{2k\pi}{n}, k\in[0,n-1]
        \end{cases}
    \]
\item[Racine complexe d'une équation du second degrée] : $az^2+bz+c=0$
    \[
        \delta^2=b^2-4ac
        \textrm{ alors }
        z=\frac{-b\pm\delta}{2a}
    \]
\item[Polynomes premiers] : Les seuls polynômes premier de $\C[X]$ sont les polynomes constants, ceux de degré 1
    et ceux de degré 2 qui n’ont pas de racine réelles
\item[Multiplicité d’une racine] : Soit $P$ un polynôme de $\C[X]$
    \[
        r\textrm{ de multiplicité }m
        \Leftrightarrow
        P(r)=P'(r)=\dots =P^{(n-1)}(r)=0
        \textrm{ et }
        P^{(m)}(r)\ne 0
    \]
\item[Partie entière d’une fraction rationnelle] :
    Soit $F = P/Q \in \C(X)$ on peut décomposer $F$ de façon unique
    tel que $F = E + \frac{P_0}{Q}$ avec, ou $P_0 = 0$ ou $deg(P_0 ) < deg(Q)$
\item[Décomposition en élément simple dans $\C(X)$] : Soit $F = P/Q$
    \\Objectif : écrire $F$ sous la forme $F = P^* + S$ où $P^*$ est un polynôme et $S$ une somme d’éléments simples :
    \\Si $deg(P ) < deg(Q)$ alors $P^* = 0$
    \\Sinon effectuer la division euclidienne
    \\Décomposer $Q$ en produit de facteur premier
    \\Règles de décomposition dont les constantes $a, b, c, d, \dots$ sont à déterminer :
    \begin{align*}
        \frac{N(x)}{(x-1)(x-2)}&=\frac{a}{x-1}+\frac{b}{x-2}\\
        \frac{N(x)}{(x-1)^3(x-2)^2}&=
            \frac{a}{x-1}
            +\frac{b}{(x-1)^2}
            +\frac{c}{(x-1)^3}
            +\frac{d}{x-2}
            +\frac{e}{(x-2)^2}\\
        \frac{N(x)}{(x-1)(x^2+1)}&=
            \frac{a}{x-1}
            +\frac{bx+c}{x^2+1}
    \end{align*}
\end{description}