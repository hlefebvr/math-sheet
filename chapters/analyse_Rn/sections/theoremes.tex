\subsection{Théorème de Stokes-Ampères}
Soit $S$ une surface de $\R^3$ et $\Gamma$ le bord de $S$ (courbe fermée), alors pour $\vect V=(P(M),Q(M),R(M))^T$ on a
\[
    \iint_S\rot\vect V
    =\oint_\Gamma Pdx+Qdy+Rdz
\]
C'est-à-dire
\[
    \mathcal T_\Gamma(\vect V)=
    \Phi_S(\rot\vect V)
\]

\subsection{Théorème de Gauss-Ostrogradski}
Soit $V$ un volume de $\R^3$ limité par une surface $\Sigma$, on a
\[
    \iiint_V\divg\vect V=
    \iint_\Sigma\vect V.\vect nd\sigma=
    \Phi_\Sigma(\vect V)
\]