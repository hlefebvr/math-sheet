\subsection{Généralités}
\begin{description}
\item[Disque ouvert de centre $A$ et de rayon $\rho$] :
    \[
        B(A,\rho)=\{M\in\R^n,||\vect{AM}<\rho||\}
    \]
\item[Limité] :
    \[
        \limite{M}{M_0}f(M)=l
        \Leftrightarrow
        \forall\varepsilon >0,
        \exists\eta > 0\tq
        \forall M\in\R^n,
        ||\vect{M_0M}||<\eta
        \Rightarrow
        |f(M)-l|<\varepsilon
    \]
\item[Continuité] :
    \[
        \limite{M}{M_0}f(M)=f(M_0)
    \]
\item[Condition suffisante de continuité] :
    \[
        \begin{cases}
            x=x_0+r\cos\theta\\
            y=y_0+r\sin\theta
        \end{cases},
        \exists\varepsilon\tq
        |f(M)-f(M_0)|<\varepsilon(r)
        \textrm{ avec }\varepsilon\overset{r\rightarrow 0}{\longrightarrow}0
        \Rightarrow
        |f(M)-l|<\varepsilon
    \]
\item[Condition suffisante de non-continuité] : S'il existe un chemin $C\tq$
    \[
        \limite{M}{M_0}f(M)\ne f(M_0)\Rightarrow f\textrm{ n'est pas continue}
    \]
\end{description}
\subsection{Dérivation}
\begin{description}
\item[Différentiabilité] : $f$ différentiable si
    \[
        f(x_0+h,y_0+h)=
        f(x_0,y_0)+Ah+Bh+\sqrt{h^2+k^2}\varepsilon(h,k)
        \textrm{ avec }
        \varepsilon\longrightarrow 0
    \]
\item[Condition suffisante de différentiabilité] : Si $f$ admet des dérivées partielles premières continues en $M_0$
    alors $f$ est différentiable en $M_0$
\item[Théorème de Schwarz] :
    \[
        \frac{\partial^2f}{\partial x\partial y},
        \frac{\partial^2f}{\partial y\partial x}
        \in C^0
        \Rightarrow
        \frac{\partial^2f}{\partial x\partial y}=
        \frac{\partial^2f}{\partial y\partial x}
    \]
\item[Dérivation de composée de fonctions] :
\begin{enumerate}
    \item $\Phi(t)=f(\alpha(t),\beta(t))$
        \[
            \Phi'(t)=
            \alpha'(t)\frac{\partial}{\partial x}
            f(\alpha(t),\beta(t))
            +
            \beta'(t)\frac{\partial}{\partial y}
            f(\alpha(t),\beta(t))
        \]
    \item $\psi(u,v)=f(a(u,v),b(u,v))$
        \begin{align*}
            \frac{\partial\psi}{\partial u}(u,v)
            &=
            \frac{\partial a}{\partial u}(u,v)
            \frac{\partial f}{\partial x}(f(a(u,v)),b(u,v))
            +
            \frac{\partial b}{\partial u}(u,v)
            \frac{\partial f}{\partial y}(f(a(u,v)),b(u,v))\\
            \frac{\partial\psi}{\partial v}(u,v)
            &=
            \frac{\partial a}{\partial v}(u,v)
            \frac{\partial f}{\partial x}(f(a(u,v)),b(u,v))
            +
            \frac{\partial b}{\partial v}(u,v)
            \frac{\partial f}{\partial y}(f(a(u,v)),b(u,v))
        \end{align*}
    \item $\zeta(x,y)=\alpha(f(x,y))$
        \begin{align*}
            \frac{\partial\zeta}{\partial x}(x,y)
            &=
            \frac{\partial f}{\partial x}(x,y)
            \alpha'(f(x,y))\\
            \frac{\partial\zeta}{\partial y}(x,y)
            &=
            \frac{\partial f}{\partial y}(x,y)
            \alpha'(f(x,y))
        \end{align*}
\end{enumerate}
\item[Différentielle] :
    \[
        df=\frac{\partial f}{\partial x}dx
        +\frac{\partial f}{\partial y}dy
    \]
\item[Formule des accroissements finis] :
    \[
        f(x_0+h,y_0+k)=
        f(x_0,y_0)
        +\frac{\partial f}{\partial x}(x_0,y_0)h
        +\frac{\partial f}{\partial y}(x_0,y_0)k
        +\sqrt{h^2+k^2}\varepsilon(h,k)
    \]
\item[Taylor à l'ordre 2] :
    \begin{align*}
        f(x_0+h,y_0+k)
        =&
        f(x_0,y_0)
        +\frac{\partial f}{\partial x}(x_0,y_0)h
        +\frac{\partial f}{\partial y}(x_0,y_0)k\\
        &+\frac{1}{2}\left(
            \frac{\partial^2f}{\partial x^2}(x_0,y_0)h^2
            +\frac{\partial^2f}{\partial y^2}(x_0,y_0)k^2
            +2\frac{\partial^2f}{\partial x\partial y}hk
        \right)
        + (h^2+k^2)\varepsilon(h,k)
    \end{align*}
\item[Condition nécessaire d'optimalité] :
    \[
        \dpartial{f}{x}(x^*,y^*)=
        \dpartial{f}{y}(x^*,y^*)=0
    \]
    Puis, repasser à Taylor :
    \[
        f(x_0+h,y_0+k)-f(x_0,y_0)=
        \frac{1}{2}\left(
            \frac{\partial^2f}{\partial x^2}(x_0,y_0)h^2
            +\frac{\partial^2f}{\partial y^2}(x_0,y_0)k^2
            +2\frac{\partial^2f}{\partial x\partial y}hk
        \right)
        + (h^2+k^2)\varepsilon(h,k)
    \]
\end{description}
\subsection{Dérivées directionnelles}
\begin{description}
\item[Définition] :
    \[
        Df(x,y)=\limite{\lambda}{0}\frac{f(x+\lambda y)-f(x)}{\lambda}
    \]
    Remarque :
    \begin{enumerate}
        \item $Df(x,\vect{e_i})=\dpartial{f}{x_i}(x)$
        \item Si $f$ est différentiable, alors $Df(x,y)=Df(x)y$
    \end{enumerate}
\item[Théorème] :
    \[
        f(x^*)\le f(x),\forall x\in\R^n
        \Leftrightarrow
        Df(x^*,y)=0, \forall y\in\R^n
    \]
\item[Existence] : Si $f$ est continue et $\limite{||x||}{\infty}f(x)=+\infty$ alors $x^*$ existe
\item[Unicité] : Si $f$ est une fonction convexe, alors $x^*$, s'il existe, est unique
\end{description}